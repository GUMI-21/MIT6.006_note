%
% 6.006 problem set 1 solutions template
%
\documentclass[12pt,twoside]{article}

\input{macros}
\newcommand{\theproblemsetnum}{1}
\usepackage{color}
\usepackage{listings}
\lstset{ %
language=C++,                % choose the language of the code
basicstyle=\footnotesize,       % the size of the fonts that are used for the code
numbers=left,                   % where to put the line-numbers
numberstyle=\footnotesize,      % the size of the fonts that are used for the line-numbers
stepnumber=1,                   % the step between two line-numbers. If it is 1 each line will be numbered
numbersep=5pt,                  % how far the line-numbers are from the code
backgroundcolor=\color{white},  % choose the background color. You must add \usepackage{color}
showspaces=false,               % show spaces adding particular underscores
showstringspaces=false,         % underline spaces within strings
showtabs=false,                 % show tabs within strings adding particular underscores
frame=single,           % adds a frame around the code
tabsize=2,          % sets default tabsize to 2 spaces
captionpos=b,           % sets the caption-position to bottom
breaklines=true,        % sets automatic line breaking
breakatwhitespace=false,    % sets if automatic breaks should only happen at whitespace
escapeinside={\%*}{*)}          % if you want to add a comment within your code
}
\title{6.006 Problem Set 1}

\begin{document}

\handout{Problem Set \theproblemsetnum}\\

\setlength{\parindent}{0pt}
\medskip\hrulefill\medskip

{\bf Name:} Zhang Wenhao

\medskip

{\bf Collaborators:} 

\medskip\hrulefill

%%%%%%%%%%%%%%%%%%%%%%%%%%%%%%%%%%%%%%%%%%%%%%%%%%%%%
% See below for common and useful latex constructs. %
%%%%%%%%%%%%%%%%%%%%%%%%%%%%%%%%%%%%%%%%%%%%%%%%%%%%%

% Some useful commands:
%$f(x) = \Theta(x)$
%$T(x, y) \leq \log(x) + 2^y + \binom{2n}{n}$
% {\tt code\_function}


% You can create unnumbered lists as follows:
%\begin{itemize}
%    \item First item in a list
%        \begin{itemize}
%            \item First item in a list
%                \begin{itemize}
%                    \item First item in a list
%                    \item Second item in a list
%                \end{itemize}
%            \item Second item in a list
%        \end{itemize}
%    \item Second item in a list
%\end{itemize}

% You can create numbered lists as follows:
%\begin{enumerate}
%    \item First item in a list
%    \item Second item in a list
%    \item Third item in a list
%\end{enumerate}

% You can write aligned equations as follows:
%\begin{align}
%    \begin{split}
%        (x+y)^3 &= (x+y)^2(x+y) \\
%                &= (x^2+2xy+y^2)(x+y) \\
%                &= (x^3+2x^2y+xy^2) + (x^2y+2xy^2+y^3) \\
%                &= x^3+3x^2y+3xy^2+y^3
%    \end{split}
%\end{align}

% You can create grids/matrices as follows:
%\begin{align}
%    A =
%    \begin{bmatrix}
%        A_{11} & A_{21} \\
%        A_{21} & A_{22}
%    \end{bmatrix}
%\end{align}

% You can include images and PDFs as follows:
% \includegraphics[width=0.5\textwidth]{img.jpg}

\begin{problems}

\problem  % Problem 1

\begin{problemparts}
\problempart % Problem 1a

we can simplify thest to 
$
f_1 = \Theta (n\log n), f_2 = \Theta ((\log n)^n), f_3 = \Theta(\log (n^6006)) 
f_4 = \Theta(6006\log n), f_5 = \theta(\log \log (6006n)),
$
so the sequence is \textcolor{blue}{$(f_5,f_3,f_4,f_1,f_2)$}.
\\ \textcolor{red}{correct}
\problempart % Problem 1b

we take logrithm to every functions, 
then $f_1=n\log 2, f_2 = n\log 6006, f_3 = 6006^(n)\log 2,\\
f_4 = 2^n \log6006, f_5 = n^2\log 6006$,
so the sequence of functions is \textcolor{blue}{$(f_1,f_2,f_5,f_4,f_3)$}.
\\ \textcolor{red}{correct}

\problempart % Problem 1c

$(\{f_2,f_5\},f_4,f_1,f_3)$
\\ \textcolor{red}{correct}

\problempart % Problem 1d

take the logrithm to functions,$(\{f_5,f_2\},f_1,f_3,f_4)$
\\ \textcolor{red}{correct}

\end{problemparts}

\newpage
\problem  % Problem 2
\begin{problemparts}
\problempart % Problem 2a

\begin{lstlisting}
    # problem1-2 a
    def reverse(D, i, k):
        if k < 2:
            return
        for j in range(k/2):
            x1 = D.delete_at(i-1+j)
            x2 = D.delete_at(i+k-1-j)
            D.insert_at(i-1+j,x2)
            D.insert_at(i+k-1-j,x1)
    # answer: use recursive
    def reverse(D, i, k):
        if k < 2:
            return # base case
        x2 = D.delete_at(i+k-1)
        x1 = D.delete_at(i)
        D.insert_at(i, x2)
        D.insert_at(i + k - 1, x1)
        reverse(D, i+1, k - 2)
\end{lstlisting}

\problempart % Problem 2b

\begin{lstlisting}
# problem1-2 b
def move(D, i, k, j):
    if k < 1:   # base case
        return
    x1 = D.delete_at(i)
    if j > i:
        j - 1
    D.insert_at(j,x1)
    if i > j:
        i = i + 1
    move(D, i, k-1, j)
\end{lstlisting}

\end{problemparts}

\newpage
\problem  % Problem 3
skip
\newpage
\problem  % Problem 4

\begin{problemparts}
\problempart % Problem 4a

1.insert\_first(x): Construct a new node as a storing x. If the double link D is empty,then link both
D.head/D.tail to a.Otherwise set b equal to L.head,then
set a's next pointer point to b and b.pre pointer point to a,then set L.head point to a. 
\\ \\
2.insert\_last(x): Same as insert\_first, Construct a new node as a sotring x, if the
double link D is empty, set D.head and D.tail point to a.
otherwise set b equal to L.tail, a.pre point to b, b.next point to a, and L.tail ponit to a.
\\ \\
3.delete\_first(): First, if linklist is empty, reutrn.If L.head.next equal None, set L.head equal None, return.
Otherwise set a equal to D.head.next,
b equal to D.head, set b.next point to None, D.head point to a.
\\ \\
4.delet\_last(): First, if linklist is empty, reutrn. If L.Tail.pre equal None, set L.Tail. equal None, return.
Otherwise set a equal to D.Tail.pre,
b equal to D.Tail, set b.pre point to None, D.tail point to a.
\problempart % Problem 4b
First, constuct a new empty double Link list $L_1$, set L1.head point to x1,
L1.tail point to x2, then L1 is the double Link list we need to return. \\
Now, remove x1 to x2 from L:\\
if x1 is L's head and x2 is not L's tail, set L.head point to x2.next, x2.next.pre point to None. \\
if x1 is L's head and x2 is L's tail, set L.head and L.tail all point to None \\
if x1 is not L's head and x2 is L's tail, set x1.pre.next point to Null, and L's tail point to x1.pre. \\
otherwise set x1.pre.next point to x2.next \& x2.next.pre point to x1.pre.

\problempart % Problem 4c

if L2.head = None, it means L2 is empty, reutrn.\\
if x.next = None, x.next = L2.head, L2.head.pre = x, then set L2.head \& L2.tail all point to None. \\
if x.next != None, set p = x.next, x.next = L2.head, L2.head.pre = x, L2.head point to None. p.pre = L.tail, L.tail.next = p, then L.tail point to None.

\problempart Submit your implementation to {\small\url{alg.mit.edu}}.
\end{problemparts}

\end{problems}

\end{document}
